\documentclass{imo_en}

\usepackage[T1]{fontenc}
\usepackage[utf8]{inputenc}
\usepackage[english]{babel}
\usepackage[english=british]{csquotes}

\usetikzlibrary{arrows, shadings} % for triangle45 arrows
\usetikzlibrary{calc} % for out/in angle of arcs in graph
\usepackage{enumerate}

% useful packages
\usepackage{amsmath,amssymb,amstext}          % math related
\usepackage{array,booktabs,dcolumn,colortbl}  % table features
\usepackage{tikz}  

% My macros
\newcommand{\Z}{\mathbb{Z}}
\newcommand{\Q}{\mathbb{Q}}
\newcommand{\R}{\mathbb{R}}
\newcommand{\N}{\mathbb{N}}
\newcommand{\PSD}{\mathcal{S}}

\newcommand{\set}[1]{\{#1\}}

% Define the things common to all exercise sheets.

\coursetitle{Compact Course Polynomial Optimization} % / \\ Algebraic Methods of Discrete Optimization}

\courseurl{https://www.mathcore.ovgu.de/TEACHING/COMPACTCOURSES/2020opt.php}
%\courseurl{} %%% No Url

\coursepeople{%
    Dr.\ Maximilian Merkert
  %
  % \texorpdfstring{\\}{, }%
  %  Other guys%
}

\courseterm{Summer 2020}
%\coursegraphic{\includegraphics[height=14mm]{logo-math-gray.pdf}}
%\coursegraphic{\includegraphics[height=14mm]{ovgu-mathcore-text.pdf}}
\coursegraphic{\includegraphics[height=14mm]{mathcore-text.pdf}}

\sheetnumber{1}
\sheetdeadline{July 22, 2020}

\begin{document}
\maketitle

%%%%%%%%%%%%%%%%%%%%%%%%%%%%%%%%%%%%%%%%%%%%%%%%%%%%%%%%%%%%%%

{\large \textbf{Exercise \thesheetnumber .1}} \bigskip 

Consider the polynomial optimization problem

$$ \inf \set{f(x) \mid x \in \R^n} $$

with $f \in \R[X]$.

Show that the following holds:
\begin{enumerate}[a)]
	\item If $n = 1$ and the infimum is finite, it will be attained at some $x \in \R^n$. 
	\item For every $n \geq 2$, there exists a polynomial $f$ such that the infimum is finite but not attained.
\end{enumerate} 

\bigskip

%%%%%%%%%%%%%%%%%%%%%%%%%%%%%%%%%%%%%%%%%%%%%%%%%%%%%%%%%%%%%%

{\large \textbf{Exercise \thesheetnumber .2}} \bigskip 

Show that if a matrix $A \in \PSD^k$ is psd, then it can be written as

$$ A = u_1 u_1^\top + \ldots + u_r u_r^\top $$

for finitely many vectors $u_1, \ldots, u_r \in \R^k$.

Can the choice of $r$ be bounded in terms of $k$?

\bigskip

%%%%%%%%%%%%%%%%%%%%%%%%%%%%%%%%%%%%%%%%%%%%%%%%%%%%%%%%%%%%%%

{\large \textbf{Exercise \thesheetnumber .3}} \bigskip 

Consider the polynomial

$$ f = 2 + X_1^2 + X_1^2 X_2^4 - 4X_1 X_2.$$

\begin{enumerate}[a)]
	\item Show that $f$ is SOS.
	\item Determine a vector $m(X)$ of monomials and a PSD matrix $Z$ with $f = {m(X)}^\top Z m(X)$.
	\item Describe, for your choice of $m(X)$, all PSD matrices $Z$ satisfying $f = {m(X)}^\top Z m(X)$.
\end{enumerate}

\bigskip

%%%%%%%%%%%%%%%%%%%%%%%%%%%%%%%%%%%%%%%%%%%%%%%%%%%%%%%%%%%%%%

{\large \textbf{Exercise \thesheetnumber .4}} \bigskip 

Consider the homogenization

$$ h(X_1, X_2, X_3):= X_3^6 - 3X_1^2 X_2^2 X_3^2 + X_1^2 X_2^4 + X_1^4 X_2^2 $$
of the Motzkin polynomial (called the \emph{Motzkin form}). Show that

\begin{enumerate}[a)]
	\item $f = h(X_1 , 1, X_3 )$ is non-negative.
	\item $f$ is not SOS.
	\item $h(X_1, 1, X_3) + c$ is SOS for some $c \in \R$.
\end{enumerate}

\bigskip

%%%%%%%%%%%%%%%%%%%%%%%%%%%%%%%%%%%%%%%%%%%%%%%%%%%%%%%%%%%%%%

%{\large \textbf{Exercise \thesheetnumber .5}} \bigskip 
%
%Show that $f \in \R[X]$ is the zero polynomial if and only if $f = 0$ on $\R^n$ .

\end{document}

