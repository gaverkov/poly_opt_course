\documentclass{imo_en}

\usepackage[T1]{fontenc}
\usepackage[utf8]{inputenc}
\usepackage[english]{babel}
\usepackage[english=british]{csquotes}

\usetikzlibrary{arrows, shadings} % for triangle45 arrows
\usetikzlibrary{calc} % for out/in angle of arcs in graph
\usepackage{enumerate}

% useful packages
\usepackage{amsmath,amssymb,amstext}          % math related
\usepackage{array,booktabs,dcolumn,colortbl}  % table features
\usepackage{tikz}  

% My macros
\newcommand{\Z}{\mathbb{Z}}
\newcommand{\Q}{\mathbb{Q}}
\newcommand{\R}{\mathbb{R}}
\newcommand{\N}{\mathbb{N}}
\newcommand{\PSD}{\mathcal{S}}

\newcommand{\set}[1]{\{#1\}}

% Define the things common to all exercise sheets.

\coursetitle{Compact Course Polynomial Optimization} % / \\ Algebraic Methods of Discrete Optimization}

\courseurl{https://www.mathcore.ovgu.de/TEACHING/COMPACTCOURSES/2020opt.php}
%\courseurl{} %%% No Url

\coursepeople{%
    Dr.\ Maximilian Merkert
  %
  % \texorpdfstring{\\}{, }%
  %  Other guys%
}

\courseterm{Summer 2020}
%\coursegraphic{\includegraphics[height=14mm]{logo-math-gray.pdf}}
%\coursegraphic{\includegraphics[height=14mm]{ovgu-mathcore-text.pdf}}
\coursegraphic{\includegraphics[height=14mm]{mathcore-text.pdf}}

\sheetnumber{2}
\sheetdeadline{July 23, 2020}

\begin{document}
\maketitle

%%%%%%%%%%%%%%%%%%%%%%%%%%%%%%%%%%%%%%%%%%%%%%%%%%%%%%%%%%%%%%

{\large \textbf{Exercise \thesheetnumber .1}} \bigskip 

The field $\R(X)$ can be ordered in more than one way.
\begin{enumerate}[a)]
	\item Try to find orderings other than the one presented in the lecture that extend the ordering of $\R$. 
	\item Can you describe all such orderings?
\end{enumerate} 

\bigskip

%%%%%%%%%%%%%%%%%%%%%%%%%%%%%%%%%%%%%%%%%%%%%%%%%%%%%%%%%%%%%%

{\large \textbf{Exercise \thesheetnumber .2}} \bigskip 

Show the following: For a real closed field $R$, the set $\sum R^2$ is the
unique ordering of $R$. Even more specifically, an element of $R$ is non-negative if and
only if it is a square $x^2$ with $x \in R$.

\bigskip

%%%%%%%%%%%%%%%%%%%%%%%%%%%%%%%%%%%%%%%%%%%%%%%%%%%%%%%%%%%%%%

{\large \textbf{Exercise \thesheetnumber .3}} \bigskip 

Let $k \in \N$. Is the positive semidefinte cone $\PSD^k_+$ semialgebraic? Is it basic semialgebraic?

\bigskip

%%%%%%%%%%%%%%%%%%%%%%%%%%%%%%%%%%%%%%%%%%%%%%%%%%%%%%%%%%%%%%

{\large \textbf{Exercise \thesheetnumber .4}} \bigskip 

Find a quantifier-free formula equivalent to

$$ F(p, q) := \exists x~[~-1 \leq x \leq 1,~ x^2 + px + q = 0~].$$

Draw a sketch of the respective semialgebraic set $S :=  \set{(p, q) \in \R^2 : F(p, q) \text { is \emph{true}} }$. Is it basic semialgebraic?

%\bigskip

%%%%%%%%%%%%%%%%%%%%%%%%%%%%%%%%%%%%%%%%%%%%%%%%%%%%%%%%%%%%%%

%{\large \textbf{Exercise \thesheetnumber .5}} \bigskip 
%
%Show that $f \in \R[X]$ is the zero polynomial if and only if $f = 0$ on $\R^n$ .

\end{document}

