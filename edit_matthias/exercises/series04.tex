\documentclass{imo_en}

\usepackage[T1]{fontenc}
\usepackage[utf8]{inputenc}
\usepackage[english]{babel}
\usepackage[english=british]{csquotes}

\usetikzlibrary{arrows, shadings} % for triangle45 arrows
\usetikzlibrary{calc} % for out/in angle of arcs in graph
\usepackage{enumerate}

% useful packages
\usepackage{amsmath,amssymb,amstext}          % math related
\usepackage{array,booktabs,dcolumn,colortbl}  % table features
\usepackage{tikz}  

% My macros
\newcommand{\Z}{\mathbb{Z}}
\newcommand{\Q}{\mathbb{Q}}
\newcommand{\R}{\mathbb{R}}
\newcommand{\N}{\mathbb{N}}
\newcommand{\PSD}{\mathcal{S}}
\newcommand{\sprod}[2]{\left\langle #1 , #2 \right\rangle}
\newcommand{\intr}{\operatorname{int}}

\newcommand{\set}[1]{\{#1\}}

% Define the things common to all exercise sheets.

\coursetitle{Compact Course Polynomial Optimization} % / \\ Algebraic Methods of Discrete Optimization}

\courseurl{https://www.mathcore.ovgu.de/TEACHING/COMPACTCOURSES/2020opt.php}
%\courseurl{} %%% No Url

\coursepeople{%
    Dr.\ Maximilian Merkert
  %
  % \texorpdfstring{\\}{, }%
  %  Other guys%
}

\courseterm{Summer 2020}
%\coursegraphic{\includegraphics[height=14mm]{logo-math-gray.pdf}}
%\coursegraphic{\includegraphics[height=14mm]{ovgu-mathcore-text.pdf}}
\coursegraphic{\includegraphics[height=14mm]{mathcore-text.pdf}}

\sheetnumber{4}
\sheetdeadline{July 27, 2020}

\begin{document}
\maketitle

%%%%%%%%%%%%%%%%%%%%%%%%%%%%%%%%%%%%%%%%%%%%%%%%%%%%%%%%%%%%%%

{\large \textbf{Exercise \thesheetnumber .1}} \bigskip 

Let $m, n \in \N$, let $K \subseteq \R^n$ and $L \subseteq \R^m$ be closed convex cones and let $A \in \R^{m \times n}, c \in \R^n$ and $b \in \R^m$. Consider the problems
	\begin{equation*}
	\alpha := \inf \set{ \sprod{c}{x} \mid x \in K, \ A x - b \in L}
	\end{equation*}
	and
	\begin{equation*}
	\beta := \sup \set{ \sprod{y}{b} \mid y \in L^\ast, \ c - A^\top y \in K^\ast}.
	\end{equation*}
	Show that the following hold:
	\begin{enumerate}[(a)]
		\item The problems satisfy weak duality. 
		\item If there exists an $x'$ with $x' \in \intr(K)$ and $A x - b \in \intr(L)$, then strong duality holds.
		\item If there exists an $y'$ with $y' \in \intr(L^\ast)$ and $c - A^\top y \in \intr(K^\ast)$, then strong duality holds.
	\end{enumerate}

\bigskip

%%%%%%%%%%%%%%%%%%%%%%%%%%%%%%%%%%%%%%%%%%%%%%%%%%%%%%%%%%%%%%

{\large \textbf{Exercise \thesheetnumber .2}} \bigskip

Show that for every $k \in \N$ one has $(\PSD_+^k)^\ast = \PSD_+^k$ (i.e., the cone $\PSD_+^k$ is self-dual).

\bigskip

%%%%%%%%%%%%%%%%%%%%%%%%%%%%%%%%%%%%%%%%%%%%%%%%%%%%%%%%%%%%%%

{\large \textbf{Exercise \thesheetnumber .3}} \bigskip 

Formulate the problem of computing the largest eigenvalue of a symmetric matrix as an SDP.

\bigskip

%%%%%%%%%%%%%%%%%%%%%%%%%%%%%%%%%%%%%%%%%%%%%%%%%%%%%%%%%%%%%%

{\large \textbf{Exercise \thesheetnumber .4}} \bigskip 

Consider the SDP
$$\inf \left\{ x~\colon~\begin{pmatrix} 0 & x \\ x & y \end{pmatrix} \ \text{psd},~x  \ge -1 \right\}. $$
	\begin{enumerate}[(a)]
		\item What is the optimal value of this problem? 
		\item What is the optimal value of its dual?
	\end{enumerate}
%\bigskip

%%%%%%%%%%%%%%%%%%%%%%%%%%%%%%%%%%%%%%%%%%%%%%%%%%%%%%%%%%%%%%

%{\large \textbf{Exercise \thesheetnumber .5}} \bigskip 
%
%Show that $f \in \R[X]$ is the zero polynomial if and only if $f = 0$ on $\R^n$ .

\end{document}

