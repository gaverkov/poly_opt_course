\section{Interior-point methods for linear and semidefinite optimization} 

In principle, one could start directly with the discussion of interior-point methods for SDP, to shorten the presentation. However, it is easier to understand the technicalities of the SDP case, if one got the idea of interior-point methods for LP first. So, we first describe general ideas, then discuss LP and then go over to SDP.

\subsection{Ways to solve constrained convex problems: a very short synopsis}

We are talking about sufficiently smooth problems here. % Methods to find locally optimal solutions for constrained and unconstrained optimization were discussed by Sebastian Sager. 
Unconstrained problems can be solved by first and second order methods (gradient descent, Newton and the many generalizations and ramifications thereof). To solve constrained problems, one basic approach is to approximate the original problem by an unconstrained problem by `deforming' the objective functions. So, we replace the original problem by a so-called auxiliary problem (with an auxiliary objective function), which is unconstrained (or as good as unconstrained).  One can distinguish between penalty and barrier methods. In barrier methods we end up with optimizing a function over an open set (which is, more or less, unconstrained optimization). In penalty methods, the auxiliary function is defined on the whole space and the optimal solution for the auxiliary problem may not be feasible (but is hopefully almost feasible) for the original problem.

Yet, another method is the ellipsoid method. In principle, it can be applied for LP, SDP and other problems and it can be used to show solvability in polynomial time, but by now it has only been of theoretical importance.

For solving SDP we are free to choose any of the methods listed above; however, so far the interior-point methods have been the most successful ones. To develop an interior-point method, we need to decide which barrier we could use and then to figure out how to compute the derivatives of the modified objective function. 

\subsection{Central path for LP with constraints $Ax \le b$}

Consider the system $A x \le b$ which we can write as $\sprod{a_i}{x} \le b_i$, so $a_1,\ldots,a_m$ are the rows of $A$. We assume that the polyhedron $P$ defined by $A x \le b$ has non-empty interior. We assume that none of the inequalities are trivial ones having the form $\sprod{0}{x} \le 0$. In this case the interior of $P$ is described by $A x < b$. We introduce the objective $f(x) = \sprod{c}{x}$. Our problem is $\inf \setcond{f(x)}{x \in P}$. For every $\mu \ge 0$ consider the auxiliary objective function 
\[
	f_\mu(x) = \sprod{c}{x} - \mu \sum_{i=1}^m \ln(b_i- \sprod{a_i}{x})
\]
The sum of the logarithms is called a barrier function. For each given $\mu> 0$, we get the auxiliary problem $\min \setcond{f_\mu(x)}{x \in \intr(P)}$. 

\begin{exercise}
	Show that, if the above assumptions on $P$ are fulfilled, and $P$ is bounded, then $f_\mu$ is strictly convex on $\intr(P)$. Show that $f_\mu$ attains its infimum on $\intr(P)$. 
\end{exercise}

Thus, if $P$ is bounded, the auxiliary problem has a unique optimal solution $x^\ast(\mu)$. The map $\mu \mapsto x^\ast(\mu)$ is a parameterization of a curve (the so-called \emph{central path}). 

The idea of central path methods is to start with $x^\ast(\mu)$ for some $\mu>0$ and then decrease $\mu$ until a desired accuracy is reached. If $\mu_2 \ge \mu_1$ and $x^\ast(\mu_2)$ is given we can use $x^\ast(\mu_2)$ as an approximation for $x^\ast(\mu_1)$, so that one can invoke various kinds of iterative methods (most notably Newton and damped Newton) that would start with $x^\ast(\mu_2)$ to determine $x^\ast(\mu_1)$. In this way, we can gradually decrease $\mu$ reaching arbitrarily small values and making our auxiliary problem arbitrarily close to the original one.

\subsection{Central path for LP with constraints $Ax=b, x \ge 0$}

Here we consider the problem in the form 
\[
	\min \setcond{\sprod{c}{x}}{A x = b, x \ge 0}.
\]
where $A \in \R^{m \times n}$, $b \in \R^m$ and $x=(x_1,\ldots,x_n)$ is the vector of decision variables.
We assume that there exists $x > 0$ with $A x =b$. In that case we can introduce the auxiliary  problem $\inf \setcond{f_\mu(x)}{A x= b, x > 0}$ with the auxiliary objective 
\[
	f_\mu(x) = \sprod{c}{x} - \mu \sum_{j=1}^n \ln x_j
\]
where $\mu>0$ is a parameter. For the function $f_\mu$ defined on $\setcond{x}{A x =b, x >0}$ one can establish analogous properties as for the function from the previous section (it is strictly convex and the optimum is attained, when $\setcond{x}{A x=b,x \ge 0}$ is bounded). 

Since $f_\mu$ is convex, a necessary and a sufficient condition for $x >0$ to be optimal for $f_\mu(x)$ is that $Ax =b$ is fulfilled and $\nabla f_\mu(x)$ is orthogonal to the space $A x =b$. Since $\setcond{x}{A x =b}$ coincides with $\ker(A)$ up to translations, we get the condition $\nabla f_\mu(x) \in \ker(A)^\perp$. Since $\ker(A)^\perp = \im(A^\top)$, the condition can be rewritten as $\nabla f_\mu(x) \in \im(A^\top)$. The latter means that $\nabla f_\mu(x)$ is in the linear hull of the columns of $A$. The condition $\nabla f_\mu(x) \in \im(A^\top)$ is exactly the KKT condition, phrased in geometric terms. Noticing that $\nabla f_\mu(x) = c+ \mu (1/x_1,\ldots,1/x_n)$, we can now formulate the KKT conditions as
\begin{align*}
	A x & = b,
	\\ 	c+ \mu (1/x_1,\ldots,1/x_n) & = A^\top y
\end{align*}
where $y \in \R^m$. Introducing the vector $s= \mu (1/x_1,\ldots,1/x_n)$, we can rewrite the latter conditions as 

\begin{align}
	A x & = b, \label{primal:eq}
	\\ A^\top y - s & = c \label{dual:eq}
	\\ (s_1 x_1,\ldots,s_n x_n) & = \mu (1,\ldots,1) \label{mu:slack}
	\\ x, s & > 0. \label{strict:pos}
\end{align}

All the equations are linear apart from the equations $s_i x_i = \mu$. 


	For solving the system \eqref{primal:eq}--\eqref{strict:pos} we can use the Newton method. If we have an approximate solution $x, y, s$, which satisfies $A x = b$ and $A^\top y -s  =c$ exactly and satisfies the conditions $x_i s_i = \mu$ approximately, we can try to determine a better solution in the form $x + \Delta x, y + \Delta y, s + \Delta s$. For $\Delta x, \Delta y$ and $\Delta s$, we obtain the conditions $A \Delta x = 0$, $A^\top \Delta y - \Delta s = 0$. By linearizing the condition $(s_i + \Delta s_i) (x_i + \Delta x_i)  = \mu$ removing the quadratic term $\Delta s_i \Delta x_i$, we arrive at $s_i \Delta x_i + \Delta s_i x_i = \mu - s_i x_i$. The variables $\Delta s_i$ can be eliminated, as we can express them via $\Delta y$. In this way we end up with a linear system of equalities in the unknowns $\Delta x, \Delta y$. The size of the system is $m+n$. If $A$ has full row rank (which we can assume wlog), then the latter system has a unique solution). We would have to take care that the update produces a solution which is still positive (so, we are on the safe side if we use damped Newton). 

	Let's estimate how well the auxiliary problem approximates the original one. In view of the boundedness assumption LP duality gives us the equality
	\[
		\alpha := \min \setcond{ \sprod{c}{x} }{x \in \R^n, \ A x = b, x \ge 0} = \max \setcond{ \sprod{b}{y}}{ y \in \R^m, \ A^\top y \le c} \in \R.
	\]
	of the optimal values of the primal and the dual problem. 
	The dual problem can also be written using the slack vector $s \ge 0$ as $A^\top y + s = c, y \in \R^m s \in \R_+^n$. We can thus see now that this version of the central path method constructs a primal/dual pair of solutions. It turns out, these solutions get nearly optimal as $\mu \to 0$. 
	
	Indeed, repeating the derivation of the weak duality, we get the estimates
	\[
		\alpha \ge \sprod{b}{y} = \sprod{A x}{y}= \sprod{x}{A^\top y} = \sprod{x}{c-s} = \sprod{x}{c} - \sprod{x}{s} = \sprod{x}{c} - n \mu \ge \alpha - n \mu.
	\]
	Hence 
	\[
		\alpha - n \mu \le \sprod{b}{y} \le \alpha 
	\]
	and 
	\[
		\alpha \le \sprod{c}{x} \le \alpha + n \mu.
	\]
	So, our solutions are optimal up to the additive tolerance $n \mu$. 

\subsection{$\ln(\det(X))$ is a good barrier for SDP}

A natural barrier we can employ for the SDP cone is $\ln(\det(X))$. Indeed, on the boundary of $\cS_+^k$ the determinant gets zero, while in the interior of $\cS_+^k$ the determinant is strictly positive. So, taking the logarithm, we create a function that goes to (minus) infinity on the boundary. This is exactly what we want. 

For the auxiliary problem to be convex, the respective barrier needs to be a convex (or concave) function. We'll verify that $\ln(\det(X))$ is concave in $X$. Furthermore, in order to be able to apply first and second order methods for solving the auxiliary problem, we need to determine the first and the second derivatives of $\ln(\det(X))$. Note that the gradient and the Hesse matrix in an abstract Euclidean space can be defined analogously to the concrete space $\R^n$. That is, $\nabla f(x^\ast)$ satisfies $f(x) = f(x^\ast) + \sprod{\nabla f(x^\ast)}{x-x^\ast} + o(\|x-x^\ast\|)$, where $o(x-x^\ast)$ is a function with $o(\|x-x^\ast\|) / \|x-x^\ast\| \to 0$ for $x \to x^\ast$ and the `Hesse-Matrix' $\nabla^2 f(x^\ast)$ is a self-adjoint operator satisfying $f(x) = f(x^\ast) + \sprod{\nabla f(x^\ast)}{x-x^\ast} + \frac{1}{2} \sprod{\nabla^2 f(x^\ast) (x-x^\ast)}{x-x^\ast} + o(\|x-x^\ast\|^2)$. 


\begin{exercise}[Root of a psd matrix]
	\label{matrix:root}
	Show that for every $X \in \cS_+^k$, there exists a uniquely defined matrix $X^{1/2} \in \cS_+^k$ satisfying $(X^{1/2})^2 = X$. 
\end{exercise}
\begin{solution}
	We can write $\R^n = V_1 \oplus \ldots \oplus V_m$, where $V_1,\ldots,V_m$ are eigenspaces of $X$ to the $m$ pairwise distinct eigenvalues $\lambda_1,\ldots,\lambda_m \ge 0$. Thus, we can choose $X^{1/2}$ that sends $v \in V_i$ to $\lambda_i^{1/2} v$ on $V_i$. It remains to show uniqueness. Let $Y \in \cS_+^k$ satisfy $Y^2 = X$. We consider eigenspaces $W_1,\ldots,W_s$ of $Y$ to its eigenvalues $\mu_1,\ldots,\mu_s$. Because of $Y^2=X$ we see that $W_1,\ldots,W_s$ are eigenspaces of $X$ to the eigenvalues $\mu_1^2,\ldots,\mu_s^2$. So, we get $s=m$ and up to permutation of indices we must have $\mu_i^2 = \lambda_i$ for all $i \in [m]$.
\end{solution}

\begin{exercise}
	\label{interchange}
	For all $A, B \in \R^{n \times n}$ one has $\tr(AB)=\tr(BA)$. 
\end{exercise}
\begin{solution}
	Use the formula for the product of matrices and the formula for the trace (as the sum of diagonal entries) and see that the left and the right hand side are just the same. 
\end{solution}

\begin{proposition}[{see, for example, Renegar, \S 1.2}]
	The function $f : \intr(\cS_+^k) \to \R$ given by $f(X) := \ln (\det (X))$ is strictly concave and infinitely differentiable. Its gradient is 
	\[
		\nabla f(X) = X^{-1}
	\]
	and its `Hesse-Matrix' is the operator
	\[
		(\nabla^2 f(X)) (U) = X^{-1} U X^{-1}
	\]
	These two formulas can also be summarized as the second-order Taylor expansion 
	\[
		f(X+U) = f(X) + \sprod{X^{-1}}{U} + \frac{1}{2} \sprod{X^{-1} U X^{-1} }{U} + o(\|U\|^2), \qquad \text{as} \ U \to 0
	\]
	for every $X \in \intr(\cS_+^k)$. (The norm $\|U\|$ is the norm with respect to the scalar product that we fixed in $\cS^k$; it is the so-called Frobenius norm)
\end{proposition}
\begin{proof}
	Note that $\det(X)$ is a polynomial function with respect to the components of $X$ and it is strictly positive on $\intr(S^k)$. The function $\ln$ is infinitely differentiable on $\R_{>0}$. So, $f$ is infinitely differentiable. 
	
	\emph{Strict concavity of $f$}. It suffices to check the strict concavity on each line. This means that we check the strict concavity of $\phi(t) = \ln \det(X+ t U)$ for every $X \in \cS_+^k \setminus \{0\}$ and every $U \in \cS^k \setminus \{0\}$. We do this by showing that the second derivative is strictly negative
	
	\begin{align*}
		\phi(t) & = \ln ( \det(X^{1/2} X^{1/2} + t U) )
		\\ & =  \ln \bigl(\det(X^{1/2} (I + t X^{-1/2} U X^{-1/2}) X^{1/2}) \bigr)
		\\ & = \ln \bigl( \det(X^{1/2}) \det(I + t X^{-1/2} U X^{-1/2}) \det(X^{1/2}) \bigr)
		\\ & = \ln \bigl( \det(X) \det(I+ t X^{-1/2} U X^{-1/2}) \bigr)
	\end{align*}
	where $X^{-1/2} = (X^{1/2})^{-1}$. Let's use the notation  $\Tilde{U}:=X^{-1/2} U X^{-1/2}$. This gives
	\begin{align*}
		\phi(t) = \ln(\det(X)) + \ln \det(I+ t \Tilde{U}).
	\end{align*}
	It suffices to compute the derivative of $\ln \det(I + t \Tilde{U})$. $\Tilde{U}$ is symmetric, and we introduce its eigenvalues $\Tilde{\lambda}_i$ (taking into account the multiplicities). Then $1 + t \Tilde{\lambda}_i$ are the eigenvalues of $I + t \Tilde{U}$. Since $I + t \Tilde{U}$ (for $X + t U \in \intr(\cS_+^k)$) is positive definite, we have $1 + t \Tilde{\lambda}_i > 0$. Since the determinant of a symmetric matrix is a product of all the eigenvalues, we obtain 
	\[
		\phi'(t) = \sum_{i=1}^n \parti{}{t} \ln(1+ t \Tilde{\lambda}_i) = \sum_{i=1}^n \frac{\Tilde{\lambda}_i}{1 + t \Tilde{\lambda}_i}.
	\]
	and the second derivative is 
	\[
		\phi''(t) = - \sum_{i=1}^n \frac{ \Tilde{\lambda}_i^2}{ (1 + t \Tilde{\lambda}_i)^2}
	\]
	We have $\phi''(t) \le 0$. We cannot have $\phi''(t) = 0$, because this would imply $\Tilde{\lambda}_i = 0$ for all $i$. But then we'd have $\Tilde{U} = 0$ and by this $U = 0$.
	
	For computing the gradient of $f$ we can use the above computation for  $\phi'(t)$. We get
	\[
		\phi'(0)= \sum_{i=1}^n \Tilde{\lambda}_i = \tr(X^{-1/2} U X^{-1/2}) = \tr( X^{-1/2} X^{-1/2} U) = \tr(X^{-1} U) = \sprod{X^{-1}}{U}.
	\]
	So, we get the `directional' derivative
	\[
		\parti{f}{U}(X) = \sprod{X^{-1}}{U}
	\]
	which gives $\nabla f(X) = X^{-1}$. 
	
	For the derivation of the `Hesse-Matrix' we can use the above equation for $\phi''(t)$. We get
	\begin{align*}
		\phi''(0) = - \sum_{i=1}^n \Tilde{\lambda}_i^2 = \tr( \Tilde{U}^2) & = \tr(X^{-1/2} U X^{-1/2} X^{-1/2} U X^{-1/2} ) 	
		\\ & = \tr(X^{-1/2} U X^{-1} U X^{-1/2}) 
		\\ & = \tr(U X^{-1} U X^{-1} )  & & \text{(by Exercise~\ref{interchange})}
		\\ & = \sprod{U}{X^{-1} U X^{-1}}.
	\end{align*}
\end{proof}


\subsection{Interior-point methods for SDP}

I rely more or less on Chapter~6 of \cite{Gaertner:Matousek:2012}.

Let's consider the SDP

\[
	\inf \setcond{ \sprod{C}{X} }{X\in \cS_+^k, \ \sprod{A_i}{X}=b_i ~ \forall i \in [m]}.
\]

We assume that the problem is strictly feasible, that is, there exists $X \in \intr(\cS_+^k)$ with $\sprod{A_i}{X} = b_i$ for all $i \in [m]$. We introduce a parameter $\mu > 0$ and introduce the auxiliary problem
\[
	\inf \setcond{ \sprod{C}{X} - \mu \ln(\det(X)) }{X \in \intr(\cS_+^k), \ \sprod{A_i}{X} = b_i \ \forall i \in [m]}.
\]

Our auxiliary objective function $f_\mu(X)$ is strictly convex and the the infimum is actually a minimum, because $f_\mu(X) \to \infty$ when $X$ approaches the boundary of $\cS_+^k$. Due to the strict convexity, the minimum is unique. By KKT-condition, the gradient of $f_\mu$ is orthogonal to the affine space $\setcond{X \in \cS^k}{\sprod{A_i}{X}=b_i \ \forall i \in [m]}$. Thus, we can write the gradient of $f_\mu$ as a linear combination of the `normal vectors' $A_i$ of the hyperplanes $\sprod{A_i}{X}=b_i$. So, we arrive at the condition 

\[
	C - \mu X^{-1} = \sum_{i=1}^m y_i A_i
\]

We introduce the matrix $S = \mu X^{-1}$, for which the identity $S X = \mu I$ holds. Summarizing, we arrive at the system

\begin{align}
	X, S & \in \intr(\cS_+^k),  \ y_1,\ldots,y_m \in \R \label{eq:prim:dual:var} \\
	\sprod{A_i}{X} & = b_i \label{eq:lin:eq} \qquad \forall i \in [m]\\
		S + \sum_{i=1}^m y_i A_i & = C,  \label{eq:prim:dual:lin:eq}\\ 
	S X & = \mu I. \label{eq:mu:slackness}
\end{align}

As in the case of LP, one can see that $y_1,\ldots,y_m$ satisfying the above conditions is a feasible solution of the dual problem: 

\[
	\sup \setcond{y_1 b_1 + \cdots + y_m b_m}{ C - \sum_{i=1}^m y_i A_i \in \cS_+^k}.
\]

Note also that $S X  = \mu I$ is a kind of modified complementary slackness. 

Again, \eqref{eq:prim:dual:var}--\eqref{eq:mu:slackness} can be solved using iterative methods (say, damped Newton). Assume that we have $X,S,y_1,\ldots,y_m$ that satisfy all the above constraints exactly, except for the constraint $S X = \mu I$, which is only satisfied approximately.

For updating the current approximation $X,S,y_1,\ldots,y_m$ to a new $X+\Delta{X}, S + \Delta{S}, y_1 + \Delta{y_1},\ldots,y_m + \Delta{y_m}$, we plug in the new approximation and linearize the last constraint with respect to the $\Delta{S}$ and $\Delta{X}$:

\begin{align}
	\sprod{A_i}{\Delta X} & = 0, \qquad \forall i \in [m] \label{eq:for:delta:1}
	\\ \Delta{S} + \sum_{i=1}^m \Delta{y_i} A_i & = 0,\label{eq:for:delta:2}
	\\ \Delta{S} X + S \Delta{X} & = \mu I - S X \label{eq:for:delta:3}
\end{align}

That is the point, where the theory of interior-point methods for SDP starts. We've got a system of linear equations. Solving such equations is linear algebra, but the size is large and we want to solve the system approximately. 

Various approaches exist. One approach is to first give up the symmetry of $\Delta{X}$ and then do an update using  $(\Delta{X} + \Delta{X}^\top) /2$ rather than $\Delta{X}$. With this approach, from \eqref{eq:for:delta:3} we get 

\[
	\Delta{X} = \mu S^{-1} - X - S^{-1} \Delta{S} X.
\]
We plug in $\Delta{S}$ from \eqref{eq:for:delta:2} and get 
\[
	\Delta{X} = \mu S^{-1} - X + \sum_{j=1}^m \Delta{y}_j S^{-1} A_j X.
\]

Inserting this expression for $\Delta{X}$ into \eqref{eq:for:delta:1}, which can be written as \mbox{$\tr(A_i \Delta{X}) = 0$}, we arrive at the equation 

\[
	\sum_{j=1}^m \tr(A_i S^{-1} A_j X) \Delta{y}_j = \tr((X - \mu S^{-1}) A_i).
\]

This is a $m \times m$ linear system. It can be shown that, if $A_1,\ldots,A_m$ are linearly independent vectors of the space $\cS^k$, then the matrix of the system is symmetric positive definite (so that there exists a unique solution). The method of finding the non-symmetric $\Delta X$ and then symmetrizing works well in practice (though polynomial time bounds on the number of iterations are obtained using different, more involved methods, references can be found in \cite{Gaertner:Matousek:2012}). Regarding the complexity: Determination of $\Delta{X}$ can be viewed as solving $k$ linear equations of size $k \times k$. So, $O(k^4)$ operations. Solving the linear system for the $\Delta{y}_1,\ldots,\Delta{y}_m$ requires $O(m^3)$ operations. Note that $m$ can be as large as $O(k^2)$. Thus, we end up with $O(m^3) = O(k^6)$. When $m$ is not large, say $m=O(k)$, we still have $O(k^4)$ in one iteration of the Newton step. In the applications related to polynomial optimization, both $m$ and $k$ are big. So, one is forced to work with really huge matrices.