\documentclass{imo_en}

\usepackage[T1]{fontenc}
\usepackage[utf8]{inputenc}
\usepackage[english]{babel}
\usepackage[english=british]{csquotes}

\usetikzlibrary{arrows, shadings} % for triangle45 arrows
\usetikzlibrary{calc} % for out/in angle of arcs in graph
\usepackage{enumerate}

% useful packages
\usepackage{amsmath,amssymb,amstext}          % math related
\usepackage{array,booktabs,dcolumn,colortbl}  % table features
\usepackage{tikz}  

% My macros
\newcommand{\Z}{\mathbb{Z}}
\newcommand{\Q}{\mathbb{Q}}
\newcommand{\R}{\mathbb{R}}
\newcommand{\N}{\mathbb{N}}
\newcommand{\PSD}{\mathcal{S}}

\newcommand{\set}[1]{\{#1\}}

% Define the things common to all exercise sheets.

\coursetitle{Compact Course Polynomial Optimization} % / \\ Algebraic Methods of Discrete Optimization}

\courseurl{https://www.mathcore.ovgu.de/TEACHING/COMPACTCOURSES/2020opt.php}
%\courseurl{} %%% No Url

\coursepeople{%
    Dr.\ Maximilian Merkert
  %
  % \texorpdfstring{\\}{, }%
  %  Other guys%
}

\courseterm{Summer 2020}
%\coursegraphic{\includegraphics[height=14mm]{logo-math-gray.pdf}}
%\coursegraphic{\includegraphics[height=14mm]{ovgu-mathcore-text.pdf}}
\coursegraphic{\includegraphics[height=14mm]{mathcore-text.pdf}}

\sheetnumber{3}
\sheetdeadline{July 24, 2020}

\begin{document}
\maketitle

%%%%%%%%%%%%%%%%%%%%%%%%%%%%%%%%%%%%%%%%%%%%%%%%%%%%%%%%%%%%%%

{\large \textbf{Exercise \thesheetnumber .1}} \bigskip 

Show that the slice of

$$\bar{P}_{2,2}(\Delta) := \set{ (a, b, c) \in \R^3 \mid f := ax^2 + bxy + cy^2 \geq 0 \text{ on } \Delta}$$
by the hyperplane $a + b + c = 1$ is unbounded.

\bigskip

%%%%%%%%%%%%%%%%%%%%%%%%%%%%%%%%%%%%%%%%%%%%%%%%%%%%%%%%%%%%%%

{\large \textbf{Exercise \thesheetnumber .2}} \bigskip

Show the following: Let $X_1,\ldots, X_n, Y_1, \ldots, Y_n$ be indeterminates, and let $E^n_+$ and $E^n_-$ , respectively, be the set of all vectors $e \in \set{-1, 1}^n$ with an even resp. odd number of entries equal to $-1$. Then
$$ X_1 \cdot \ldots \cdot X_n \pm Y_1 \cdot \ldots \cdot Y_n = \frac{1}{2^{n-1}} \sum_{e \in E_\pm^n} \prod_{i=1}^n (X_i + e_i Y_i).$$

In particular, $X_1 \cdot \ldots \cdot X_n \pm Y_1 \cdot \ldots \cdot Y_n$ belong to the semiring generated by $X_1 + Y_1, \ldots, X_n + Y_n, X_1 - Y_1, \ldots, X_n - Y_n$.

\bigskip

%%%%%%%%%%%%%%%%%%%%%%%%%%%%%%%%%%%%%%%%%%%%%%%%%%%%%%%%%%%%%%

{\large \textbf{Exercise \thesheetnumber .3}} \bigskip 

Show that if a polynomial $f$ is positive on $\set{a \geq 0}$, then
$$(1 + h)f = 1 + g$$
holds for some $g, h \in \mathcal{P}(a)$.
\vspace{1\baselineskip}

Hint: Use the polynomial version of Farkas lemma: If $-1 \in \mathcal{P}(a)$, then every polynomial is in $\mathcal{P}(a)$.

%\bigskip
%
%%%%%%%%%%%%%%%%%%%%%%%%%%%%%%%%%%%%%%%%%%%%%%%%%%%%%%%%%%%%%%%
%
%{\large \textbf{Exercise \thesheetnumber .4}} \bigskip 
%
%Construct an algebraic certificate for the fact that there is no stable set of size $>1$ in a complete graph on $3$ vertices. Use the following system to model the stable sets in the above problem:
%
%$$x^2_1 - x_1 = 0,x^2_2 - x_2 = 0,x^2_2 - x_2 = 0$$
%$$-x_1 -x_2 + 1 \geq 0, -x_1 -x_3 + 1 \geq 0, -x_2 -x_3 + 1 \geq 0.$$

%\bigskip

%%%%%%%%%%%%%%%%%%%%%%%%%%%%%%%%%%%%%%%%%%%%%%%%%%%%%%%%%%%%%%

%{\large \textbf{Exercise \thesheetnumber .5}} \bigskip 
%
%Show that $f \in \R[X]$ is the zero polynomial if and only if $f = 0$ on $\R^n$ .

\end{document}

