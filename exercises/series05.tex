\documentclass{imo_en}

\usepackage[T1]{fontenc}
\usepackage[utf8]{inputenc}
\usepackage[english]{babel}
\usepackage[english=british]{csquotes}

\usetikzlibrary{arrows, shadings} % for triangle45 arrows
\usetikzlibrary{calc} % for out/in angle of arcs in graph
\usepackage{enumerate}

% useful packages
\usepackage{amsmath,amssymb,amstext}          % math related
\usepackage{array,booktabs,dcolumn,colortbl}  % table features
\usepackage{tikz}  

% My macros
\newcommand{\Z}{\mathbb{Z}}
\newcommand{\Q}{\mathbb{Q}}
\newcommand{\R}{\mathbb{R}}
\newcommand{\N}{\mathbb{N}}
\newcommand{\PSD}{\mathcal{S}}

\newcommand{\set}[1]{\{#1\}}

% Define the things common to all exercise sheets.

\coursetitle{Compact Course Polynomial Optimization} % / \\ Algebraic Methods of Discrete Optimization}

\courseurl{https://www.mathcore.ovgu.de/TEACHING/COMPACTCOURSES/2020opt.php}
%\courseurl{} %%% No Url

\coursepeople{%
    Dr.\ Maximilian Merkert
  %
  % \texorpdfstring{\\}{, }%
  %  Other guys%
}

\courseterm{Summer 2020}
%\coursegraphic{\includegraphics[height=14mm]{logo-math-gray.pdf}}
%\coursegraphic{\includegraphics[height=14mm]{ovgu-mathcore-text.pdf}}
\coursegraphic{\includegraphics[height=14mm]{mathcore-text.pdf}}

\sheetnumber{5}
\sheetdeadline{July 28, 2020}

\begin{document}
\maketitle

%%%%%%%%%%%%%%%%%%%%%%%%%%%%%%%%%%%%%%%%%%%%%%%%%%%%%%%%%%%%%%

{\large \textbf{Exercise \thesheetnumber .1}} \bigskip 

Get the Matlab toolboxes YALMIP (\url{https://yalmip.github.io}) and SeDuMi (\url{http://sedumi.ie.lehigh.edu}) running on your notebook. You may also use Octave (\url{https://www.gnu.org/software/octave/}) as a free alternative to Matlab.

\vspace{0.5\baselineskip}
Test your setup by
\begin{enumerate}[a)]
	\item computing an SOS representation of the polynomial $f = 2 + X_1^2 + X_1^2 X_2^4 - 4X_1 X_2$ from Exercise 1.3.
	\item confirming part c) of Exercise 1.4 by finding $c \in \R$ such that $$h(X_1, 1, X_3) + c:= X_3^6 - 3X_1^2 X_3^2 + X_1^2 + X_1^4 +c$$ is SOS.
\end{enumerate} 

\bigskip

%%%%%%%%%%%%%%%%%%%%%%%%%%%%%%%%%%%%%%%%%%%%%%%%%%%%%%%%%%%%%%

{\large \textbf{Exercise \thesheetnumber .2}} \bigskip

Consider the polynomials $f, g_1, g_2, g_3 \in \R[X_1,X_2]$ with
$$f=-X_1^4 - X_2^4 - 2 X_1^2 X_2^2 + 2 X_1^2 X_2 + 2 X_1 X_2^2 + 6 X_1^2 - 22 X_1 X_2 + 6 X_2^2 + 6 X_1 + 10 X_2 - 5 $$  
and $$g=(g_1, g_2, g_3)=(X_1-\frac{1}{2}, X_2-\frac{1}{2}, 1 - X_1 X_2)$$

Compute an algebraic certificate showing that $f \geq 0$ on $\set{g \geq 0}$.

\bigskip

%%%%%%%%%%%%%%%%%%%%%%%%%%%%%%%%%%%%%%%%%%%%%%%%%%%%%%%%%%%%%%

{\large \textbf{Exercise \thesheetnumber .3}} \bigskip 

Consider the following integer quadratic programming formulation for the problem of finding a cut of maximum cardinality in an undirected graph $G=(V,E)$:
\begin{equation}
\max \sum_{(i,j) \in E} \frac{1 - x_i x_j}{2} \quad \textit{s.t. } x_i \in \set{-1,1} \quad \forall i \in V. \label{maxcut} \tag{MAXCUT}
\end{equation}
Formulate the following relaxation as an SDP and test it on several instances of your choice:
$$\max \sum_{(i,j) \in E} \frac{1 - x_i \cdot x_j}{2} \quad \textit{s.t. } x_i \in \R^n,~\|x_i\|_2^2 = 1 \quad \forall i \in V,$$
where $n \in \N$. \\
What is the worst approximation ratio for the optimum of \eqref{maxcut} you observed?

%\bigskip
%
%%%%%%%%%%%%%%%%%%%%%%%%%%%%%%%%%%%%%%%%%%%%%%%%%%%%%%%%%%%%%%%
%
%{\large \textbf{Exercise \thesheetnumber .4}} \bigskip 
%
%Construct an algebraic certificate for the fact that there is no stable set of size $>1$ in a complete graph on $3$ vertices. Use the following system to model the stable sets in the above problem:
%
%$$x^2_1 - x_1 = 0,x^2_2 - x_2 = 0,x^2_2 - x_2 = 0$$
%$$-x_1 -x_2 + 1 \geq 0, -x_1 -x_3 + 1 \geq 0, -x_2 -x_3 + 1 \geq 0.$$

%\bigskip

%%%%%%%%%%%%%%%%%%%%%%%%%%%%%%%%%%%%%%%%%%%%%%%%%%%%%%%%%%%%%%

%{\large \textbf{Exercise \thesheetnumber .5}} \bigskip 
%
%Show that $f \in \R[X]$ is the zero polynomial if and only if $f = 0$ on $\R^n$ .

\end{document}

